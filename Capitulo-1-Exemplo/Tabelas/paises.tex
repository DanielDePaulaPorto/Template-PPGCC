\begin{table}[!ht]

%   float placement
%----------------------
%-h means here: Place the figure in the text where the figure environment is written, if there is enough room left on the page
%-t means top: Place it at the top of a page.
%-b means bottom: Place it at the bottom of a page.
%-p means page: Place it on a page containing only floats, such as figures and tables.
%-! allows to ignore certain parameters of LaTeX for float placement, for example:
%
%	\topfraction: maximal portion of a page (or column resp., here and below), which is allowed to be used by floats at its top, default 0.7
%	\bottomfraction: maximal portion of a page, which is allowed to be used by floats at its bottom, default value 0.3
%	\textfraction: minimal portion of a page, which would be used by body text, default value 0.2
%	\floatpagefraction: minimal portion of a float page, which has to be filled by floats, default value 0.2. This avoids too much white space on float pages.
%	topnumber: maximal number of floats allowed at the top of a page, default 2
%	bottomnumber: maximal number of floats allowed at the bottom of a page, default 1
%	totalnumber: maximal number of floats allowed at whole page, default 3

\setlength{\arrayrulewidth}{2\arrayrulewidth}  % line thickness
\setlength{\tabcolsep}{4pt} % spacing between columns
\centering

\caption{Exemplo de uso de tabelas}
\label{tablePaises}

%\fontencoding{T1}
%\fontfamily{\rmdefault}
\fontseries{m}
\fontshape{n}
%\fontsize{7.75}{8.5}
\selectfont
	\begin{tabular}{lclc}
		\toprule
		Country           & Number & Country         & Number \\ \midrule
		Germany        & 8          & USA          & 8          \\
		Brazil         & 7          & India        & 7          \\
		United Kingdom & 7          & Spain        & 6          \\
		Portugal       & 5          & Colombia     & 4          \\
		Ireland        & 4          & Israel       & 4          \\
		Italy          & 4          & Mexico       & 4          \\
		Switzerland    & 4          & Turkey       & 4          \\
		Japan          & 3          & Netherlands  & 3          \\
		Norway         & 3          & Belgium      & 2          \\
		Argentina      & 1          & Austria      & 1          \\
		China          & 1          & France       & 1          \\
		Malaysia       & 1          & Saudi Arabia & 1          \\
		Sweden         & 1          &              &            \\ \bottomrule
	\end{tabular}
\end{table}

